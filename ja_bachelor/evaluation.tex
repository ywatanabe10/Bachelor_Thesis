\chapter{今後の課題}
\section{より多様なモデルの作成}
本研究の実験では多くの状況においてある程度の数のデータを取るようにした。
しかし一般家庭や公共施設において誘導する際に様々な状況に対応するためには、より多くのモデルのパターンが必要となるだろう。
これに関しては、あらゆるパターンに対応できるようにモデルデータを充実させるか、与えられたマップに対してモデルを学習により
形成するといった手法が考えられる。

\section{モデルを利用した誘導}
本研究ではモデルを図示するツールにより誘導経路の修正を行うことが出来ることを示した。
だが理想的には、モデルを用いて与えられたマップに対して最適な誘導経路を作成する事が出来ると良い。
また人間にぶつからないようにする、人間が誘導されているか確認する、といった事のために
人間の位置を検出することが出来るデバイスが必要となるだろう。
具体的にはレーザーレンジファインダやカメラを用いた画像認識を用いれば良いと考えられる。
