\chapter{序論}

\par
公共施設やオフィスにおいて人間をサポートするために,移動式の案内ロボットが一部ですでに導入されている.
また,ホテルなどでドアボーイの代わりに荷物の運搬を行うロボットも製作されている.
このようにロボットの移動により,人間を誘導する,人間のタスクの物理的負荷を軽減する,といった幅広く実用性のあるサービスを提供できる.
こういった移動式ロボットを実現するために近年,ホイールを等角度間隔で3個以上配置した移動機構(全方位移動機構)を持つ車両である全方位台車の設計や動作制御に関する研究が行われている.
図\ref{omnitruck}に代表的な全方位台車を示す.
こういった全方位台車は並進移動や回転が出来て小さなスペースでも素早く移動できるので,今後一般家庭や公共施設にも搬送などの用途で普及すると思われる.
そのように人間のいる環境で全方位台車が上手く働くには,全方位台車によって人間の注意を誘導することが出来る必要がある.
なぜなら人間のタスクを移動によりサポートするに当たって,人間に全方位台車を追従させたり特定の方向を指示するからである.
例えば荷物を置き場まで運ぶタスクにおいては,人間が全方位台車に荷物を積み,移動する全方位台車を人間が追従し,全方位台車の指定する方向の置き場へ人間が荷物を降ろすといったプロセスとなる.

\begin{comment}
\begin{figure}[!h]
\begin{center}

\includegraphics[width=5cm,height=5cm]{omniwheel.eps}
\caption{全方位台車の例}
\label{omnitruck}
\end{center}
\end{figure}
\end{comment}

\par
ロボットにより人間の注意を誘導する研究は多く行われている.
Wolframらは,図\ref{RHINO}のようにレーザーレンジファインダ,ソナー,ステレオカメラを備えたロボット「RHINO」による美術館での誘導の研究を行った \cite{Wolfram}.
RHINOにはGUIが付いており,ユーザーの指定した展示物の方へ移動し案内をする.
Marenらは図\ref{Alpha}のヒューマノイドロボット「Alpha」を用い,腕を使った方向指示による展示物への注意の誘導を行った\cite{Maren}.
また,石井らは,図\ref{KEITA}の受付案内ロホット「KEITA」を用いて対話者に自発的に話しかけ案内を行った\cite{Isi}.
「KEITA」は全方位移動機構を持ち,移動および手を用いたジェスチャにより人を誘導する.

\begin{comment}
\begin{figure}[!h]
\begin{center}

\includegraphics[width=5cm,height=5cm]{RHINO.eps}
\caption{RHINOの画像}
\label{RHINO}
\end{center}
\end{figure}
\end{comment}

\begin{comment}
\begin{figure}[!h]
\begin{center}

\includegraphics[width=5cm,height=5cm]{Alpha.eps}
\caption{Alphaの画像}
\label{Alpha}
\end{center}
\end{figure}
\end{comment}

\begin{comment}
\begin{figure}[!h]
\begin{center}

\includegraphics[width=5cm,height=5cm]{KEITA.eps}
\caption{KEITAの画像}
\label{KEITA}
\end{center}
\end{figure}
\end{comment}

\par
ところが,全方位台車によって人間の誘導を行う場合,全方位台車と誘導される人間の位置関係のモデルが必要となる.
例えば,全方位台車が注意を向けさせたい物に対してこの位置にこの角度で存在したら,人間はこの位置にこの角度で立つよう誘導されるという関係のモデルである.
全方位台車はこのようなモデルを元に人間を想定した位置や角度へ誘導することが可能になる.
\par
ロボットにより人間の注意を誘導する既存研究において,全方位移動機構を用いていないものは人間の位置と角度を同時に誘導することが出来ていない.
そのような研究における誘導は移動と回転を別々に行わなければいけないので誘導の効率が低く,また並進が出来ないことにより移動自体の効率も低い.
また全方位移動機構を持つロボットによる研究でも,腕による方向指示のような誘導以外の用途に用いることの出来ない汎用性の低いデバイスによる情報提示を用いている.
将来的に普及するであろう搬送用の全方位台車において人間の誘導を行うためには,全方位台車の自由な移動と回転という移動能力および汎用性の高いデバイスのみによる誘導が可能なシステムが必要である.

\par
本研究では「全方位台車を用いたユーザーの位置・向き誘導モデル」を提案する.
このモデルを用いることにより,全方位台車の自由な移動および回転によって人間の位置と向きを誘導することが可能である.
モデル作成にあたって,全方位台車により人間が他の人間を誘導する動きを再現させ,その際の全方位台車及び人間の位置関係や向きを分析する,といった実験を行った.
具体的には,まず人間に位置および向きの誘導を行わせその動きのデータをモーションキャプチャで取得した.次にそのデータから位置と向きの情報を抽出しそれらの動きを全方位台車に再現させ,全方位台車による案内を被験者に受けさせた.
実験における全方位台車と被験者の位置および向きの関係を調べ,この位置と角度に全方位台車がいれば人はこの位置と角度に誘導されるというようにモデル化した.
本研究では「全方位台車を用いたユーザーの位置・向き誘導モデル」を作成し,モデルを図示するツールにより全方位台車の誘導経路の修正を実現した.
