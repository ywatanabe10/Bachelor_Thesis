\chapter{序論}

\par
対話システムは,発話解釈と発話生成の両方において発展を遂げており,我々の生活に浸透しつつある.しかし,対話システムとの対話は不自然さやストレスを与ることも少なくない.本研究の目的は,人間に不自然さやストレスを与えることが無い,人間と長期的に共存可能な対話システムの実現である.

\par
対話システムと人間との対話では,対話システムが対話相手の心的状態を推定し,それを考慮した対話を行うことが重要である・人間は,気分が落ち込んでいる対話相手の発話に対してネガティブな発話解釈をしたり,励ましの言葉をかけるように,対話相手の心的状態によって相手の発話の解釈を変えたり,自身の発話の内容を変えている.また,相手が知らないことについて詳しく説明したり,相手が好むことについて話を掘り下げる.対話システムにおいても,人間と同様に対話相手の心的状態に合わせて,発話解釈を変えたり,自身の発話の内容を変えることで,より自然な対話を実現することができる.つまり,対話相手に合わせた臨機応変な発話解釈や発話生成により対話における不自然さやストレスをなくすためには,対話相手の心的状態を推定することが重要である.

\par
人間の心的状態を推定する研究は,人間の行動情報から心的状態を推定する研究と発話情報から心的状態を推定する研究が存在する.人間の行動情報から心的状態を推定する研究は,代表的には,Baker et al.の研究がある.Baker et al.は,環境の状態と環境中を移動する人間の行動や観測状況,信念をベイズ推論に適用し,環境中を移動する人間の信念と欲求を推定している.発話情報から心的状態を推定する研究では,代表的に高橋らの研究がある.発話から得られた事象を信念と捉え,考えられる欲求の候補を生成し,尤もらしい欲求を基に発話者の意図を推定する.また,心的状態を考慮することによる検索精度の向上ために,検索クエリへの入力から人間の心的状態を推定する研究も存在する.

\par
従来研究では,行動情報のみから人間の心的状態を推定する研究や発話情報のみから人間の心的状態を推定する研究は存在した.しかし,行動情報と発話情報の両方から人間の心的状態を推定する研究はない.従来研究における心的状態の推定は,行動のみが観測される場合や発話のみが観測される場合には有効であるが,行動と発話の両方が観測される場合において,行動による発話の解釈の変化や,発話による行動の解釈の変化を捉えることができない.つまり,行動情報と発話情報の相互作用を考慮した心的状態の推定ができていない.

\par
本研究では,行動情報と発話情報の両方から人間の心的状態を推定するシステムMIoMを提案する.MIoMは,行動情報と発話情報の両方をベイズの定理に適用し,人間の心的状態を推定する.MIoMは,人間の心的状態の推定において,心的状態を一つに決め付けるのではなく,同時に複数保持し,やり取りの中でその可能性を動的に変えていく.MIoMは,行動情報と発話情報をもとに,複数保持した心的状態の可能性が大きく変動したタイミングまたは一定の時間,変動しなかったタイミングで人間に質問を提示し,質問の応答を発話情報として心的状態の推定に反映することで,発話情報による行動の解釈の変化や人間の行動による発話情報の解釈の変化を捉え,行動情報と発話情報の相互作用を考慮した心的状態の推定が可能となる.

\par
本論文の構成は以下の通りである.第二章では,関連研究においてどのように人間の心的状態を推定していたかを述べる.第三章では,人間の行動情報と発話情報から心的状態を推定するシステムMIoMを提案する.第四章では,MIoMと単一の情報のみから心的状態を推定するシステムを用いて実験的に評価し,第五章では評価結果について考察する.第六章では,MIoMにおける今度の課題について述べる.最後に,第七章で本論文を締めくくる.
