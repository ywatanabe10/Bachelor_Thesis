\chapter{考察}
\par
本実験の結果,MIoMは信念と欲求の推定の両方においてUIoM (action)およびUIoM (utterance)よりも強い相関を示した.信念推定および欲求推定においてMIoMがUIoM (aciton)やUIoM (utterance)より強い相関を示した要因の一つとして,MIoMが行動情報と発話情報の両方を信念と欲求の推定に反映していることが考えられる.本実験における設定では,行動情報と発話情報の両方が観測される設定であり,発話情報から行動情報の解釈が変わったり,行動情報から発話情報の解釈が変わることがあった.例えば,図\ref{fig:ex_env}において,発話情報によってTruck1とTruck2のどちらに向かっているかの解釈が変わることがある.また,Truck1とTruck2のどちらを望んでいるかを特定することができない曖昧な発話情報を行動情報によって補完することもある.MIoMは,発話情報による行動情報の解釈の変化や行動情報による発話情報の解釈の変化を捉え,行動情報と発話情報の相互作用を推定に反映することでUIoMの推定結果を上回ったと考えられる.MIoMの推定により,信念と欲求の推定において行動情報と発話情報を両方用いることが有効であると考えられる.

\par
また,本実験では3つの推定システムにおいて欲求推定の相関が信念推定の相関より強いことが示された.そこで本実験で用いた3つの推定システムにおいて欲求推定が信念推定よりも強い相関を示した要因を考える.実験参加者による推定結果を分析したところ,欲求推定では行動情報と発話情報の両方が大きく影響していたが,信念推定では発話情報の影響が大きいことがわかった.例えば,図\ref{fig:ex_env}において,Truck1は観測しているがTruck2を観測していない状況でTruck1からは遠ざかりTruck2に向かう状況を考える.この時,Truck2に向かっているという行動情報は信念の推定に活用することは困難であり,Truck1での食事よりもTruck2での食事を好んでいる可能性が高いという欲求の推定にのみ大きく影響することが考えられる.そのため,実験参加者による信念の推定では発話情報からの影響が大きくなったと考えられる.
また,発話情報は欲求を問うものが多かったため,信念推定に大きな影響を与えることが困難であったと考える.本実験では,表\ref{tab:q_a}の応答を発話情報として採用している.表\ref{tab:q_a}からわかるように,発話情報は欲求に関する内容である.そのため,発話情報が欲求推定に大きな影響を与えることができても信念推定においては大きな影響を与えることができないと考えれられる.このような理由から,信念推定において信念を問う発話情報を増やすことが有効であると考えられる.
