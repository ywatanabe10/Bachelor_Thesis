\chapter{考察}
\par
本実験の結果,MIoMは信念と欲求の推定の両方においてUIoM (action)およびUIoM (utterance)よりも強い相関を示した.信念推定および欲求推定においてMIoMがUIoM (aciton)やUIoM (utterance)より相関が強い要因の一つとして,MIoMが行動情報と発話情報の両方を信念と欲求の推定に反映していることが考えられる.本実験における設定では,行動情報と発話情報の両方が観測される設定であり,発話情報から行動情報の解釈が変わったり,行動情報から発話情報の解釈が変わることがあった.例えば,図\ref{fig:ex_env}において,発話情報によってTruck1とTruck2のどちらに向かっているかの解釈が変わることがある.また,行動情報によって曖昧な発話情報が補完されることもある.このような,発話情報による行動情報の解釈の変化や行動情報による発話情報の解釈の変化を捉え,信念と欲求の推定において行動情報と発話情報を両方用いることが有効であると考えられる.

\par
次に,本実験で用いた3つの推定システムにおいて,欲求推定の相関が信念推定の相関より強い要因について考える.実験参加者による推定結果を分析したところ,欲求推定では行動情報と発話情報の両方が大きく影響していたが,信念推定では発話情報の影響が大きいことがわかった.また,発話情報は欲求を問うものが多かったため,信念推定に大きな影響を与えることが困難であったと考える.このような理由から,信念推定において,信念を問う発話情報を増やすことが有効であると考えられる.
