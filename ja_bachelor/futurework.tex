\chapter{今後の課題}

\section{実世界情報による心的状態の推定}
本実験では,人工的なシミュレーションにおける人間の行動と発話から信念と欲求を推定している.行動は上,下,左,右の4方向への移動とし,発話は事前に設定した質問群の中から選択されるため,特定の条件における推定を行っている.しかし実世界では,行動や発話は多種多様であり,意図しない行動や発話が生じることも少なくない.VR機器を使用し,実際に人間に行動および対話をさせることでより実戦に即した状況における行動情報および発話情報を取得し,人間の信念や欲求の推定に活用するようにMIoMを拡張する.

\section{発話情報の検討}
% 本実験では,行動$a_t$および発話$u_t$をそれぞれの生起確率$P(a_t)$,$P(u_t)$に変換した.$P(a_t)$は$k$番目のパーティクルが持つ信念$b_t^k$と欲求$d^k$から行動$a_t$が起こる確率として定義した.$P(a_t)$は,信念$b_t^k$と欲求$d$の関係と学生と屋台との距離を基に計算を行っている.

% 埋め込みモデルの検討
% 発話の重要後抽出
% 文章扱い

MIoMにより信念および欲求を推定するにあたり,発話内容の検討および発話$u_t$の生起確率$P(u_t)$の計算方法の検討が必要である.
最初に発話内容の検討について述べる.本実験では,発話内容として表\ref{tab:answer}に示す学生の応答内容を採用した.しかし,それらはいずれも欲求に関する内容であるため,欲求の推定に影響を与えることができても信念の推定に影響を与えることが困難であった.この問題を解決するには,発話内容の検討が必要である.発話内容に関して,欲求だけでなく信念に関する発話を促す質問を提示する機構をMIoMに搭載することが重要である.
MIoMでは,発話$u_t$をWord2Vecにより分散表現に変換した後,信念$b_t$と欲求$d$との類似度を基に生起確率$P(u_t)$に変換する.


\section{対話相手の発話に対する応答生成}
