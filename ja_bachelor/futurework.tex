\chapter{今後の課題}

\section{実世界情報による心的状態の推定}
本実験では,人工的なシミュレーションにおける人間の散策行動と人間の発話文から信念と欲求を推定している.散策行動は上,下,左,右の4方向への移動とし,人間の発話文は事前に設定した応答群の中から選択されており,特定の条件における推定を行っている.しかし実世界では,散策行動や人間の発話文は多種多様であり,意図しない散策行動や人間の発話文が生じることも少なくない.今後はVR機器を使用し,実際に人間に散策行動および対話をさせることで,より実世界に似た状況における散策行動情報および人間の発話情報を取得し,人間の信念や欲求の推定に活用するためにMIoM SCAINを拡張する.

\section{人間の発話情報の検討}
% 本実験では,行動$a_t$および発話$u_t$をそれぞれの生起確率$P(a_t)$,$P(u_t)$に変換した.$P(a_t)$は$k$番目のパーティクルが持つ信念$b_t^k$と欲求$d^k$から行動$a_t$が起こる確率として定義した.$P(a_t)$は,信念$b_t^k$と欲求$d$の関係と学生と屋台との距離を基に計算を行っている.

% 埋め込みモデルの検討
% 発話の重要後抽出
% 文章扱い
\par
実験参加者による信念と欲求の推定では散策行動情報よりも人間の発話情報を重視する傾向が見られた.そこで人間の発話情報の検討をすることがMIoM SCAINにおける信念と欲求の推定結果の向上に繋がると考えられる.MIoM SCAINにより信念および欲求を推定するにあたり,人間の発話内容の検討,人間の発話文の推定への活用方法の検討および人間の発話文$u_t$の生起確率$P(u_t)$の計算方法の検討が必要である.

\par
最初に人間の発話内容の検討について考える.本実験では,人間の発話内容として表\ref{tab:q_a}に示す学生の応答内容を採用した.しかし,それらはいずれも欲求に関する内容であるため,欲求の推定に影響を与えることができても信念の推定に影響を与えることが困難であった.また,実世界では条件や制約は存在せず,人間の発話内容は多種多様である.これらの問題を解決するには,人間の発話内容の検討が必要である.人間の発話内容に関して,欲求だけでなく信念に関する発話文を促す質問を提示する機構をMIoM SCAINに搭載することが必要であると考える.信念に関する発話文を促す質問を提示する機構をMIoM SCAINに搭載することで,信念に関する発話情報を取り入れることができるだけでなく,多種多様な人間の発話文を制御することが可能となる.

\par
次に人間の発話文の推定への活用方法の検討について考える.現時点におけるMIoM SCAINでは人間の発話文が観測された場合,それを単語レベルに分解し,単語ごとに生起確率を求める.例えば,``I want to eat rice.''という人間の発話文が観測された時,``I'',``want'',``to'',``eat'',``rice''の単語レベルに分解され,それぞれの単語ごとの生起確率が計算される.しかし,例えば"to"はこの文における重要度が低く,信念や欲求の推定に大きな影響を与えることがないことが予想される.また,信念や欲求の推定に逆効果をもたらす単語が推定に反映されてしまう場合も考えられる.これらの問題を解決するには,人間の発話文の推定への活用方法の検討が必要である.そこで文の重要語抽出手法の適用が考えられる.文の重要語抽出手法を用い,信念や欲求の推定に大きな影響を与える単語を抽出し,生起確率を計算し推定に活用することで,発話文による信念および欲求の推定を実現することができる.

\par
最後に人間の発話文の生起確率の計算方法の検討について考える.MIoM SCAINでは,人間の発話$u_t$をWord2Vecにより分散表現に変換した後,信念$b_t$と欲求$d$との類似度を基に生起確率$P(u_t)$に変換する.しかし単語同士の類似度比較では不十分な場合がある.例えば``Japanese food''と``rice''の類似度を計算する場合を考える.Word2vecでは1単語同士の類似度を計算することは可能だが,この場合のような2単語と1単語の類似度計算を行うことができない.その結果,``Japanese''と``rice''の類似度を計算することになる.しかし,``Japanese''のみでは食事に関する意味が損なわれてしまい,人間が行う類似度比較とは異なることが考えられる.この問題を解決するためには,単語埋め込みモデルの検討が必要である.単語数の異なるテキスト間の類似度を計算することができる単語埋め込みモデルを使用することにより,人間が行う類似度比較に近づけることができる.したがって単語埋め込みモデルの検討により,人間の発話文の生起確率の計算方法を改善することができると考える.


\section{対話システムの発話生成}
% 推定に有利な発話を促す発話生成
% 推定結果を考慮した発話生成
% 発話生成のタイミング

\par
MIoM SCAINは人間の散策行動情報と人間の発話情報を活用した信念と欲求の推定に止まっていた.しかしMIoM SCAINにおける最終目標は,信念や欲求を考慮した対話を行うことである.そこで,推定した信念と欲求を考慮し,MIoM SCAINが発話文を生成する機構を取り入れることが必要である.この機構により,人間の信念と欲求を考慮した発話を生成することで,対話において人間に与える不自然さやストレスを軽減することができ,使いやすい対話システムの実現へ近づくことができると考える.また,この機構により人間の信念と欲求を考慮したタイミングで発話を提示することが可能となる.その結果,発話の内容だけでなく発話のタイミングによっても,対話において人間に与える不自然さやストレスを軽減することができると考える.

\par
また,信念と欲求の推定に大きな影響を与える人間の発話を促す発話を生成することも重要である.今後,より実践的な状況における散策行動情報と人間の発話情報から信念と欲求を推定するにあたり,多種多様な人間の発話が観測されることが予想される.MIoM SCAINはランダムな発話生成を行うのではなく,人間の発話に条件を設けたり,制約をかけることができる発話を生成することが必要であると考える.
