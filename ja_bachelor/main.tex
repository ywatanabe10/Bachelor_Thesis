%
%     卒業論文 (2004. 1)
%                         Authorized by Jun MUKAI
%
\documentclass[12pt]{jreport}
\usepackage{b_thesis}
\usepackage{ascmac}
\usepackage{deepsection}
\usepackage{drafts}
\usepackage[dvipdfmx]{graphicx, color}
\usepackage{here}
\usepackage{tabverb}
\usepackage{tabularx}
\usepackage{array}
\usepackage{moreverb}
\usepackage{fancybox}
\usepackage{comment}
\usepackage{algorithm}
\usepackage{algorithmic}
\usepackage{indentfirst}
\usepackage{multirow}

\pagestyle{headings}
%\pagestyle{drafts}

\topmargin=0.0truecm
\oddsidemargin=0.1truecm
\evensidemargin=0.1truecm
\textwidth=16cm
\textheight=22.5cm
\newcommand{\figurewidth}{14cm}

\title{行動情報と発話情報の組み合わせによる\\信念と欲求の逐次的推測モデルの検討}

\author{渡邊 悠太}                   % あなたのお名前
\teachera{今井 倫太 教授}
\teacherb{}
\teacherc{}
\authorname{渡邊 悠太}               % あなたのお名前
\date{}

\course{情報工学}
\id{61720736}                       % あなたの学籍番号

\begin{document}
\baselineskip = 20pt
\bibliographystyle{my_jalpha}
\nocite{*}
\maketitle                  %表紙がいらないときはコメントアウト

%
%     論文要旨
%
%
\jabst{
\hspace*{1zh}
\par
 本研究の目的は,人間にストレスを与えることの無い,人間と長期的に共存可能な対話システムの実現である.対話システムが人間の心的状態を考慮せず発話解釈や発話生成を行うと,相手に不自然さやストレスを与えることがあるため,対話相手に合わせて臨機応変に発話解釈や発話生成を行うことが必要である.対話相手に合わせた対話を行うためには,人間の心的状態を推定することが重要である.人間の心的状態を推定する研究では,人間の行動から心的状態を推定する研究や人間の発話から心的状態を推定する研究がある.しかし既存研究では,行動情報と発話情報の両方を用いて心的状態を推定することができていないため,発話による行動の解釈の変化や行動による発話の解釈の変化を捉えることができていない.

\par
 本論文では,行動情報と発話情報の両方を活用した心的状態推定システムMultimodal Inference of Mind SCAIN (MIoM SCAIN)を提案する.MIoMは,行動情報と発話情報の両方を活用して心的状態を推定することで,発話による行動の解釈の変化や行動による発話の解釈の変化を捉える.本研究では,独自で作成したデータセットを利用し,MIoMによる心的状態の推定を行い,行動情報と発話情報の両方を活用した心的状態の推定が有効であるかを評価する実験を行った.本実験の結果,MIoMの推定性能が単一情報のみから心的状態を推定するシステムの推定性能を上回り,行動情報と発話情報の両方を活用した心的状態の推定が有効であることが示された.
}
\makejabstract
     %論文要旨
\newpage

\setcounter{page}{1}
\pagenumbering{roman}

\tableofcontents     %目次
\listoffigures       %図目次
\listoftables        %表目次
\newpage

\setcounter{page}{1}
\pagenumbering{arabic}

\baselineskip = 20pt


\begin{table}[htb]
  \begin{center}
  \caption{人間による推定と推定モデルの相関}
  \begin{tabular}{l|c|c} \hline
    \multirow{2}{*}{モデル}&\multicolumn{2}{c}{相関}\\\cline{2-3}
    & \hspace{10pt} 信念 \hspace{10pt} & \hspace{10pt} 欲求 \hspace{10pt} \\ \hline
    UIoM(action)&0.124&0.419\\
    UIoM(utterance)&0.216&0.494\\
    MIoM(action + utterance)&0.244&0.549 \\\hline
  \end{tabular}
\end{center}
\end{table}


%%
%% 論文構成.ここは適宜変更する
%%
\chapter{序論}

\par
対話システムは,発話解釈と発話生成の両方において発展を遂げており,我々の生活に浸透しつつある.しかし,対話システムとの対話は不自然さやストレスを与ることも少なくない.本研究の目的は,人間に不自然さやストレスを与えることが無い,人間と長期的に共存可能な対話システムの実現である.

\par
対話システムと人間との対話では,対話システムが対話相手の心的状態を推定し,それを考慮した対話を行うことが重要である・人間は,気分が落ち込んでいる対話相手の発話に対してネガティブな発話解釈をしたり,励ましの言葉をかけるように,対話相手の心的状態によって相手の発話の解釈を変えたり,自身の発話の内容を変えている.また,相手が知らないことについて詳しく説明したり,相手が好むことについて話を掘り下げる.対話システムにおいても,人間と同様に対話相手の心的状態に合わせて,発話解釈を変えたり,自身の発話の内容を変えることで,より自然な対話を実現することができる.つまり,対話相手に合わせた臨機応変な発話解釈や発話生成により対話における不自然さやストレスをなくすためには,対話相手の心的状態を推定することが重要である.

\par
人間の心的状態を推定する研究は,人間の行動情報から心的状態を推定する研究と発話情報から心的状態を推定する研究が存在する.人間の行動情報から心的状態を推定する研究は,代表的には,Baker et al.の研究がある.Baker et al.は,環境の状態と環境中を移動する人間の行動や観測状況,信念をベイズ推論に適用し,環境中を移動する人間の信念と欲求を推定している.発話情報から心的状態を推定する研究では,代表的に高橋らの研究がある.発話から得られた事象を信念と捉え,考えられる欲求の候補を生成し,尤もらしい欲求を基に発話者の意図を推定する.また,心的状態を考慮することによる検索精度の向上ために,検索クエリへの入力から人間の心的状態を推定する研究も存在する.

\par
従来研究では,行動情報のみから人間の心的状態を推定する研究や発話情報のみから人間の心的状態を推定する研究は存在した.しかし,行動情報と発話情報の両方から人間の心的状態を推定する研究はない.従来研究における心的状態の推定は,行動のみが観測される場合や発話のみが観測される場合には有効であるが,行動と発話の両方が観測される場合において,行動による発話の解釈の変化や,発話による行動の解釈の変化を捉えることができない.つまり,行動情報と発話情報の相互作用を考慮した心的状態の推定ができていない.

\par
本研究では,行動情報と発話情報の両方から人間の心的状態を推定するシステムMIoMを提案する.MIoMは,行動情報と発話情報の両方をベイズの定理に適用し,人間の心的状態を推定する.MIoMは,人間の心的状態の推定において,心的状態を一つに決め付けるのではなく,同時に複数保持し,やり取りの中でその可能性を動的に変えていく.MIoMは,行動情報と発話情報をもとに,複数保持した心的状態の可能性が大きく変動したタイミングまたは一定の時間,変動しなかったタイミングで人間に質問を提示し,質問の応答を発話情報として心的状態の推定に反映することで,発話情報による行動の解釈の変化や人間の行動による発話情報の解釈の変化を捉え,行動情報と発話情報の相互作用を考慮した心的状態の推定が可能となる.

\par
本論文の構成は以下の通りである.第二章では,関連研究においてどのように人間の心的状態を推定していたかを述べる.第三章では,人間の行動情報と発話情報から心的状態を推定するシステムMIoMを提案する.第四章では,MIoMと単一の情報のみから心的状態を推定するシステムを用いて実験的に評価し,第五章では評価結果について考察する.第六章では,MIoMにおける今度の課題について述べる.最後に,第七章で本論文を締めくくる.

\chapter{関連研究}
\section{行動情報から心的状態を推定する研究}
\par
行動情報から心的状態を推定する研究として,人間の心的状態理論Theory of Mind(ToM)をベイズ推論を用いてモデル化するBToM\cite{baker2011bayesian}が存在する.BToMは,環境の状態や人間の心的状態を部分的に観測可能なマルコフ決定過程(POMDP)として表し,環境における人間の行動をベイズ推論に適用することで,人間が観測できていない環境領域についての信念と欲求を共同推定する.人間の信念と欲求を共同推定することで,信念と欲求の相互作用を捉えることができる.また,運転状況から心的状態を推定する研究\cite{darwish2020learning}も存在する.運転者がどのような行動をとることを意図しているかを推定するために,交通状況を部分的に観測可能なマルコフ決定過程として表し,運転者の心的状態をBToMを使ってモデル化している.その結果,自動運転において運転者の意図に沿った動作を実現することができる.また,作業用ロボットに行動情報と心的状態の関係を適用した研究\cite{inbook}も存在する.人間の作業員のアシスタントとして,人間の意図推定できるロボットを導入することで,効率的かつ安全な作業を行うことを目的としている.人間の行動から意図を推定する.人間の心的状態の移り変わりも考慮している.

\section{言語情報から心的状態を推定する研究}
\par
言語情報から心的状態を推定する研究として,発話から信念や欲求,意図を推定する研究\cite{10.1007/978-3-642-02481-8_4}がある.最初に,発話から事象を抽出し,信念として捉える.信念を基に,欲求の候補を複数生成する.生成された欲求の候補に対し,情緒生起手法\cite{2002}を適用し,それぞれの尤度を算出する.願望候補のうち,最も尤度が高い願望を選択し,それを基に意図生成を行う.

\section{単一情報による心的状態推定の問題点}
\par
上記の研究は,いずれも行動情報と発話情報の一方のみを心的状態の推定に用いるため,行動のみを観測できる場合や発話のみを観測できる場合には有効であるが,行動と発話の両方を観測できる場合においては不十分である.
% 例えば,上司と部下がともに行動する人間の心的状態を推定する場面を考える.上司の意思により行動内容が決定された場合,行動情報のみから心的状態を推定する研究では,部下の心的状態を推定することは難しい.
行動によって発話の解釈が変わったり,発話によって行動の解釈が変わることは少なくない.

例えば,

行動情報と発話情報の一方のみの単一情報による心的状態の推定では,行動による発話の解釈の変化や発話による行動の解釈の変化を捉えることができず,行動情報と発話情報の相互作用を考慮して心的状態を推定することができない.

\chapter{Multimodal Inference of Mind SCAIN}
\par
本論文では,行動情報と発話情報の両方を活用し,信念と欲求の推定を行うシステムMultimodal Inference of Mind SCAIN (MIoM SCAIN)を提案する.MIoM SCAINは,人間の行動,発話および人間が存在する環境の状態を基に信念と欲求を推定する.行動情報と発話情報の両方を信念と欲求の推定に活用することで,発話による行動の解釈の変化や行動による発話の解釈の変化を捉え,行動情報と発話情報の相互作用を考慮して信念と欲求を推定する.

\par
MIoM SCAINは,環境の状態や人間の心的状態を部分的に観測可能なマルコフ決定過程として表す.また,信念と欲求の推定値はそれぞれの候補に対し尤度が与えられたパーティクルフィルタとして表され,信念と欲求の組み合わせを一意に決定するのではなく同時に複数保持し,時刻が経過する度に各パーティクルの尤度を更新していく.各時刻における人間の信念や行動,発話および人間が存在する環境の状態をベイズ推論に適用し,環境中で人間が観測できていない部分についての信念と欲求を逐次的に推定する.


\section{アルゴリズム}

\par
MIoM SCAINにおける推定処理の流れをを図\ref{fig:sys_arc}に示す.
\begin{figure}[htbp]
  \begin{center}
    \includegraphics[scale=0.7]{./bt1.pdf}
    \caption{MIoM SCAINによる推定処理}
    \label{fig:sys_arc}
  \end{center}
\end{figure}
図\ref{fig:sys_arc}に示すように,MIoM SCAINは時刻$t$における人間の行動$a_t$,発話$u_t$および環境の状態$s_t$から信念と欲求の確率を出力する.MIoM SCAINは信念$b_t$と欲求$d$の組み合わせとその尤度$L$を持つパーティクルフィルタとして表現され,$a_t,u_t,s_t$とパーティクル$k$が持つ信念$b_t^k$と欲求$d^k$を基に尤度$L^k$が更新される.ここで,尤度$L^k$は次のように表すことができる.
\begin{equation}
  \begin{split}
  \label{pf}
  L^k=P(b_t^k,d^k|s_{1:t},a_{1:t-1},u_{1:t-1})
  \end{split}
\end{equation}
ここで,$u_{1:t-1}$は,時刻$1$から時刻$t-1$までの人間の発話履歴,$P(b_t,d|s_{1:t},a_{1:t-1},u_{1:t-1})$は,$s_{1:t},a_{1:t-1}およびu_{1:t-1}$から計算される$b_t$と$d$の確率である.

\par
図\ref{fig:miom}にMIoM SCAINにおけるベイズ推論を表現するベイジアンネットワークを示す.$s_t, a_t, u_t$は観測値,$o_t, b_t, d$は確率変数として扱う.
\begin{figure}[htbp]
  \begin{center}
    \includegraphics[scale=0.73]{./miom.pdf}
    \caption{MIoM SCAINにおけるベイズ推論を表現するベイジアンネットワーク}
    \label{fig:miom}
  \end{center}
\end{figure}
MIoM SCAINにおけるベイズ推論では,BToMと同様に時刻$t$における環境の状態$s_{t}$を基に人間の観測状況$o_{t}$が決まる.また,$o_{t}$を基に人間の信念$b_{t}$が決まり,$b_{t}$と人間の欲求$d$から人間の行動$a_{t}$が決まる.それに加え,MIoM SCAINでは$b_t$と$d$から人間の発話$u_t$が決まる.また$a_{t}$が起こることにより,環境の状態は$s_{t+1}$に変化し,人間の観測状況,信念,行動および発話が再び計算される.MIoM SCAINでは各パーティクルの尤度である式(\ref{pf})を計算することを目的としている.ベイズの定理と図\ref{fig:miom}における変数間の条件付き独立性より,以下の式が成り立つ.
\begin{equation}
  \begin{split}
  \label{eq_miom}
  L^k&=P(b_t^k,d^k|s_{1:t},a_{1:t-1},u_{1:t-1})\\
  &\propto P(b_t^k,d^k,s_{1:t},a_{1:t-1},u_{1:t-1})\\
  &= \sum_{b_{t-1}^k,o_t}P(b_t^k,d^k,s_{1:t},a_{1:t-1},u_{1:t-1},b_{t-1}^k,o_t)\\
  &= \sum_{b_{t-1}^k,o_t}P(b_t^k|d^k,s_{1:t},a_{1:t-1},u_{1:t-1},b_{t-1}^k,o_t)\cdot P(d^k,s_{1:t},a_{1:t-1},u_{1:t-1},b_{t-1}^k,o_t)\\
  &= \sum_{b_{t-1}^k,o_t}P(b_t^k|b_{t-1}^k,o_t)\cdot P(o_t|d^k,s_{1:t},a_{1:t-1},u_{1:t-1},b_{t-1}^k)\\
  &\hspace{5cm} \cdot P(d^k,s_{1:t},a_{1:t-1},u_{1:t-1},b_{t-1}^k)\\
  &= \sum_{b_{t-1}^k,o_t}P(b_t^k|b_{t-1}^k,o_t)\cdot P(o_t|s_t)\cdot P(s_t|s_{t-1},a_{t-1})\\
  &\hspace{3cm} \cdot P(a_{t-1}|b_{t-1}^k,d^k)\cdot P(u_{t-1}|b_{t-1}^k,d^k)\cdot P(b_{t-1}^k,d^k,s_{1:t-1},a_{1:t-2},u_{1:t-2})\\
  &\propto \sum_{b_{t-1}^k,o_t}P(b_t^k|b_{t-1}^k,o_t)\cdot P(o_t|s_t)\cdot P(s_t|s_{t-1},a_{t-1})\\
  &\hspace{3cm} \cdot P(a_{t-1}|b_{t-1}^k,d^k)\cdot P(u_{t-1}|b_{t-1}^k,d^k)\cdot P(b_{t-1}^k,d^k|s_{1:t-1},a_{1:t-2},u_{1:t-2})\\
  \end{split}
\end{equation}
ここで,$P(b_t^k|b_{t-1}^k,o_t)$は人間の観測状況$o_t$によってパーティクル$k$の信念$b_t^k$が更新される確率,$P(o_t|s_t)$は環境の状態$s_t$において人間が観測状況$o_t$を得る確率,$P(s_t|s_{t-1},a_{t-1})$は環境の状態$s_{t-1}$において人間が行動$a_{t-1}$を起こした時に環境の状態が$s_{t}$になる確率,$P(a_{t-1}|b_{t-1}^k,d^k)$はパーティクル$k$が信念$b_{t-1}^k$,欲求$d^k$を持っている時に行動$a_{t-1}$を起こす確率,$P(u_{t-1}|b_{t-1}^k,d^k)$はパーティクル$k$が信念$b_{t-1}^k$,欲求$d^k$を持っている時に発話$u_t$を起こす確率,$P(b_{t-1}^k,d^k,s_{1:t-1},a_{1:t-2},u_{1:t-2})$は時刻$t-1$におけるパーティクル$k$の尤度である.式(\ref{eq_miom})より,$L^k$は初期値$P(b_1,d,s_1,a_0,u_0)$を決めて順次更新する計算により求めることができる.また,$L^k$は$P(b_t^k|b_{t-1}^k,o_t)$,$P(o_t|s_t)$,$P(s_t|s_{t-1},a_{t-1})$,$P(a_{t-1}|b_{t-1}^k,d^k)$および$(u_{t-1}|b_{t-1}^k,d^k)$の乗算として表すことができる.

\section{実装}
\par
% それぞれの生起確率がどのように計算されるかを記載
MIoM SCAINでは,各パーティクルにおいて$P(b_t^k|b_{t-1}^k,o_t)\cdot P(o_t|s_t)$,$P(s_t|s_{t-1},a_{t-1})$,$P(a_{t-1}|b_{t-1}^k,d^k)$および$P(u_{t-1}|b_{t-1}^k,d^k)$と前時刻のパーティクルの尤度を乗算することにより,信念と欲求の尤度を更新する.$P(b_t^k|b_{t-1}^k,o_t)\cdot P(o_t|s_t)$は,式(\ref{calc_b_o})により信念$b_t^k$と観測状況$o_t$を比較し計算する.
\begin{equation}
  \begin{split}
  \label{calc_b_o}
  P(b_t^k|b_{t-1}^k,o_t)\cdot P(o_t|s_t)=
  \begin{cases}
    1 & (b_t^kとo_tが一致する時) \\
    0 & (b_t^kとo_tが一致しない時)
  \end{cases}
  \end{split}
\end{equation}
$P(s_t|s_{t-1},a_{t-1})$は,環境の状態$s_{t-1}$において行動$a_{t-1}$が起こった時,$s_t$とは異なる環境の状態に状態遷移をするということはないという条件の下,式(\ref{calc_s})を適用し計算する.
\begin{equation}
  \begin{split}
  \label{calc_s}
  P(s_t|s_{t-1},a_{t-1})=1
  \end{split}
\end{equation}
$P(a_{t-1}|b_{t-1}^k,d^k)$は,式(\ref{calc_cost})によって計算される$cost$を式(\ref{calc_a})に適用することにより計算される.ここで,$r$は信念$b_{t-1}^k$と欲求$d^k$を基に設定された報酬,$x$は行動$a_{t-1}$により生じる人間と信念$b_{t-1}^k$の距離,$\gamma$は$x$の寄与を調整する変数,$\delta$は$cost$の寄与を調整する変数である.
\begin{equation}
  \begin{split}
  \label{calc_cost}
  cost=r\cdot \gamma^{x}
  \end{split}
\end{equation}
\begin{equation}
  \begin{split}
  \label{calc_a}
  P(a_{t-1}|b_{t-1}^k,d^k)=\mathrm{softmax}(cost,\delta)
  \end{split}
\end{equation}
$P(u_{t-1}|b_{t-1}^k,d^k)$は,式(\ref{calc_u})により計算される.ここで,関数similarityはWord2Vec \cite{mikolov2013efficient}により発話$u_t$を分散表現に変換した後,信念$b_{t-1}^k$と欲求$d^k$との類似度を計算する関数である.$\alpha$は$\mathrm{similarity}(u_{t-1},b_{t-1}^k)$の大きさを調整する変数,$\beta$は欲求$d^k$の強さに基づき,$\mathrm{similarity}(u_{t-1},b_{t-1}^k)$の大きさを調整する変数である.
\begin{equation}
  \begin{split}
  \label{calc_u}
  P(u_{t-1}|b_{t-1}^k,d^k)\propto (\alpha+\beta)\cdot\mathrm{similarity}(u_{t-1},b_{t-1}^k)
  \end{split}
\end{equation}

\chapter{評価}
\par
行動情報と発話情報の両方を反映した心的状態の推定が有効であることを示すため,信念および欲求の推定において,MIoMと単一情報による心的状態推定システムUnimodal Inference of Mind(UIoM)を比較する.行動情報と発話情報には,本研究で作成したデータセットを利用する.

\section{実験設定}

\begin{figure}[htbp]
  \begin{center}
    \includegraphics[]{./figure.pdf}
    \caption{本実験における環境.図中の"Student"は学生,"Truck1"および"Truck2"は屋台を開くスペース,中央の黒色部分は壁を表す.}
    \label{fig:ex_env}
  \end{center}
\end{figure}

\par
学生がアシストロボットとともに屋台で食事を買うシーンを想定する.図\ref{fig:ex_env}に本実験における環境の一例を示す.7$\times$5マスで表現される環境中に壁とTruck1およびTruck2で表される屋台を開くスペースが存在し,それぞれのスペースに日本食の屋台,イタリア料理の屋台,中華料理の屋台のいずれかが出店する.環境中の学生は移動し,アシストロボットと対話をしながら食事を購入する屋台を決める状況を考える.学生は,日本食の屋台,イタリア料理の屋台,中華料理の屋台の3種類のうち2種類が出店することは知っているが,どの屋台が出店しているかは知らないため,環境中を移動しアシストロボットと対話しながら食事を買う屋台を選ぶ.学生の行動$a_t$は上,下,左,右の4方向への移動とし,発話$u_t$はアシストロボットから提示される食事に関する質問に対する学生の応答とする.信念$b_t$は,壁により観測できていない屋台に関してどの屋台が出店していると考えているか,欲求$d$は学生が3種類のそれぞれの屋台をどの程度好むかを表す.



\section{実験手順}

\par
本実験には,本研究で作成したデータセットを利用した.本データセットには,屋台の組み合わせを表す環境設定と,その環境設定で考えられる学生の行動,アシストロボットからの質問,学生の応答が含まれる.屋台の組み合わせは日本食の屋台,イタリア料理の屋台,中華料理の屋台の2つの組み合わせとする6通り,学生の行動は上,下,左,右の4方向への移動,アシストロボットからの質問は表\ref{tab:question}に記載される4通り,学生の応答は表\ref{tab:answer}に記載される8通りである.MIoMによって行動情報と発話情報の両方を推定に活用することが有効であることを評価するために,パーティクルフィルタを用い行動情報と発話情報の一方のみを基に心的状態を推定するシステムUnimodal Inference of Mind (UIoM)を定義した.


\begin{table}[htb]
  \begin{center}
  \caption{アシストロボットからの質問と学生の応答}
  \label{tab:q_a}
  \begin{tabular}{lcc} \hline
    質問内容&\multicolumn{2}{c}{応答内容}\\\hline
    魚料理と野菜料理どちらを食べたいですか&fish&vegetable\\
    パスタと米ではどちらを食べたいですか&pasta&rice\\
    あっさりしたものと,こってりしたものどちらを食べたいですか&plain&oily\\
    辛いものと酸っぱいものではどちらを食べたいですか&spicy&sour\\\hline
  \end{tabular}
\end{center}
\end{table}



\par
30人の実験参加者に,本データセットで指定された環境設定と行動およびアシストロボットからの質問と学生の応答を提示し,環境中の学生の信念と欲求をそれぞれ7段階で推定させた.また,MIoM (action + utterance),行動情報のみを基に心的状態を推定するUIoM (action),発話情報のみを基に心的状態を推定するUIoM (utterance)の3つのシステムによって,環境中の学生の信念と欲求をそれぞれ7段階で推定した.実験参加者によって得られた推定結果とMIoMおよびUIoM によって得られた推定結果を比較し相関係数を算出した.UIoM (action)およびUIoM (utterance)の尤度は,それぞれ式(\ref{uiom_a}),式(\ref{uiom_u})で計算した.

\begin{equation}
  \begin{split}
  \label{uiom_a}
  L^k({\mathrm{action}})&= \sum_{b_{t-1}^k,o_t}P(b_t^k|b_{t-1}^k,o_t)\cdot P(o_t|s_t)\cdot P(s_t|s_{t-1},a_{t-1})\\
  &\hspace{3cm}\cdot P(a_{t-1}|b_{t-1}^k,d^k)\cdot P(b_{t-1}^k,d^k,s_{t-1},a_{t-2})
  \end{split}
\end{equation}

\begin{equation}
  \begin{split}
  \label{uiom_u}
  L^k({\mathrm{utterance}})&= \sum_{b_{t-1}^k,o_t}P(b_t^k|b_{t-1}^k,o_t)\cdot P(o_t|s_t)\cdot P(s_t|s_{t-1},a_{t-1})\\
  &\hspace{3cm}\cdot P(u_{t-1}|b_{t-1}^k,d^k)\cdot P(b_{t-1}^k,d^k,s_{t-1},u_{t-2})
  \end{split}
\end{equation}

\begin{figure}[htbp]
  \begin{center}
    \includegraphics[scale=0.6]{./interface.pdf}
    \caption{本実験に使用したインターフェース}
    \label{fig:interface}
  \end{center}
\end{figure}

\section{実験結果}

\par
表\ref{tab:cof}に,実験参加者による信念と欲求の推定結果とUIoM (action), UIoM (utterance)およびMIoMによる信念と欲求の推定結果との間の相関係数を示す.

\begin{table}[htb]
  \begin{center}
  \caption{人間による推定と推定モデルの相関}
  \label{tab:cof}
  \begin{tabular}{lcc} \hline
    \multirow{2}{*}{モデル}&\multicolumn{2}{c}{相関}\\\cline{2-3}
    & \hspace{10pt} 信念 \hspace{10pt} & \hspace{10pt} 欲求 \hspace{10pt} \\ \hline
    UIoM (action)&0.124&0.419\\
    UIoM (utterance)&0.216&0.494\\
    MIoM (action + utterance)&\bf0.244&\bf0.549 \\\hline
  \end{tabular}
\end{center}
\end{table}


\par
表\ref{tab:cof}より,信念と欲求の推定の両方において,行動情報と発話情報の両方を推定に活用するMIoMが行動情報のみを推定に活用するUIoM (action)および発話情報のみを推定に活用するUIoM (utterance)よりも強い相関を示した.また,いずれの推定システムにおいても欲求推定の相関が信念推定の相関よりも強いことがわかった.

\chapter{考察}
\par
本実験の結果,MIoMは信念と欲求の推定の両方においてUIoM (action)およびUIoM (utterance)よりも強い相関を示した.信念推定および欲求推定においてMIoMがUIoM (aciton)やUIoM (utterance)より相関が強い要因の一つとして,MIoMが行動情報と発話情報の両方を信念と欲求の推定に反映していることが考えられる.本実験における設定では,行動情報と発話情報の両方が観測される設定であり,発話情報から行動情報の解釈が変わったり,行動情報から発話情報の解釈が変わることがあった.例えば,図\ref{fig:ex_env}において,発話情報によってTruck1とTruck2のどちらに向かっているかの解釈が変わることがある.また,Truck1とTruck2のどちらを望んでいるかを特定することができない曖昧な発話情報を行動情報によって補完することもある.このような,発話情報による行動情報の解釈の変化や行動情報による発話情報の解釈の変化を捉え,信念と欲求の推定において行動情報と発話情報を両方用いることが有効であると考えられる.

\par
また,本実験では3つの推定システムにおいて欲求推定の相関が信念推定の相関より強いことがわかった.そこで本実験で用いた3つの推定システムにおいて欲求推定の相関が信念推定の相関より強くなった要因を考える.実験参加者による推定結果を分析したところ,欲求推定では行動情報と発話情報の両方が大きく影響していたが,信念推定では発話情報の影響が大きいことがわかった.例えば,図\ref{fig:ex_env}において,Truck1は観測しているがTruck2を観測していない状況でTruck1からは遠ざかりTruck2に向かう状況を考える.この時,Truck2に向かっているという行動情報から信念を推定することはできず,Truck1での食事よりもTruck2での食事を好んでいる可能性が高いという欲求の推定に大きく影響することが考えられる.そのため実験参加者による信念の推定では発話情報からの影響が大きくなったと考えられる.
また,発話情報は欲求を問うものが多かったため,信念推定に大きな影響を与えることが困難であったと考える.本実験では,表\ref{tab:question}の質問に対する表\ref{tab:answer}の応答を発話情報として採用している.表\ref{tab:question}および表\ref{tab:answer}からわかるように,発話情報は欲求に関する内容である.そのため,発話情報が欲求推定に大きな影響を与えることができても信念推定においては大きな影響を与えることができないと考えれられる.このような理由から,信念推定において信念を問う発話情報を増やすことが有効であると考えられる.

\chapter{今後の課題}

\section{実世界情報による心的状態の推定}
本実験では,人工的なシミュレーションにおける人間の行動と発話から信念と欲求を推定している.行動は上,下,左,右の4方向への移動とし,発話は事前に設定した質問群の中から選択されるため,特定の条件における推定を行っている.しかし実世界では,行動や発話は多種多様であり,意図しない行動や発話が生じることも少なくない.VR機器を使用し,実際に人間に行動および対話をさせることでより実戦に即した状況における行動情報および発話情報を取得し,人間の信念や欲求の推定に活用するようにMIoMを拡張する.

\section{発話情報の検討}
% 本実験では,行動$a_t$および発話$u_t$をそれぞれの生起確率$P(a_t)$,$P(u_t)$に変換した.$P(a_t)$は$k$番目のパーティクルが持つ信念$b_t^k$と欲求$d^k$から行動$a_t$が起こる確率として定義した.$P(a_t)$は,信念$b_t^k$と欲求$d$の関係と学生と屋台との距離を基に計算を行っている.

% 埋め込みモデルの検討
% 発話の重要後抽出
% 文章扱い

MIoMにより信念および欲求を推定するにあたり,発話内容の検討および発話$u_t$の生起確率$P(u_t)$の計算方法の検討が必要である.
最初に発話内容の検討について述べる.本実験では,発話内容として表\ref{tab:answer}に示す学生の応答内容を採用した.しかし,それらはいずれも欲求に関する内容であるため,欲求の推定に影響を与えることができても信念の推定に影響を与えることが困難であった.この問題を解決するには,発話内容の検討が必要である.発話内容に関して,欲求だけでなく信念に関する発話を促す質問を提示する機構をMIoMに搭載することが重要である.
MIoMでは,発話$u_t$をWord2Vecにより分散表現に変換した後,信念$b_t$と欲求$d$との類似度を基に生起確率$P(u_t)$に変換する.


\section{対話相手の発話に対する応答生成}

\chapter{結論}
本論文では,MIoM SCAINにより散策行動情報と人間の発話情報を活用したマルチモーダルな信念と欲求の推定について検討した.
実験では,MIoM SCAINと散策行動情報または人間の発話情報の一方のみを活用して推定を行うシステムUIoM SCAINの推定性能を比較した.
実験の結果,MIoM SCAINの推定性能がUIoM SCAINの推定性能を上回り,散策行動情報と人間の発話情報の両方を信念と欲求の推定に用いることが有効であることを示した.
今後の展望としては,より実世界に近い環境設定や三次元の行動および多種多様な発話を扱えるようにMIoM SCAINを拡張したいと考えている.

%%
%%

%
% 謝辞
%
\chapter*{謝辞}
\addcontentsline{toc}{chapter}{謝辞}
\baselineskip = 20pt

\begin{verbatimtab}
	本研究を進めるにあたり、研究の機会及び貴重なご意見を頂きました、
			慶應義塾大学理工学部 今井 倫太 教授
に深く感謝致します。


	論文の査読をして頂き、細部にわたって御意見を頂きました
			理工学研究科修士課程1年 大竹 七勢 氏
			理工学研究科修士課程2年 里形 理興 氏
に厚く御礼申し上げます。

  
	実験に御協力頂いた被験者の方々に心より御礼申し上げます。



最後に日頃から御指導、御協力下さいました今井研究室の皆様に心より感謝いたします。


							令和3年1月
\end{verbatimtab}
    %謝辞

\bibliography{reference}     %参考文献

\end{document}
