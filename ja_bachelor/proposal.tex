\chapter{提案}\label{chap:proporsal}

\par
本研究では「全方位台車を用いたユーザーの位置・向き 誘導モデル」を提案する.
このモデルによって全方位台車の動きと誘導される人間の位置と方向の関係がわかる.
またこのモデルを用ると,全方位台車の自由な移動と回転および汎用性の高いデバイスのみにより人間の位置と向きが誘導できる.
よって今後一般家庭や公共施設に普及するであろう搬送用の全方位台車にも役立つモデルである.
\par
本研究のモデルは,全方位台車による展示物の案内の実験を行いその結果を分析することで作成する.
本研究では全方位台車として三菱電機特機システム(株)の小型全方位ロボットを用いる.
また位置座標取得にOptiTrack社のNaturalPoint(モーションキャプチャシステム)を用いる.
具体的には以下の手順で実験分析を行う.

\begin{enumerate}
\item 実験手順
\begin{enumerate}
\item 人間の案内動作における座標および角度データを取得する
\item 上記のデータの動きを全方位台車に再現させ,被験者に案内を受けさせる
\end{enumerate}
\item 分析手順
\begin{enumerate}
\item 全方位台車が展示物の案内を行っている際の全方位台車および人間の位置と角度の関係を調べる
\item 上記を複数の展示物および被験者で行い,全方位台車と人間の位置および角度の関係のモデル化を行う
\end{enumerate}
\end{enumerate}



\par
このようにして作成したモデルに従い,案内動作に対して予測される人間の位置と角度を図示するツールを作成する.
このツールに人間の案内動作を入力することにより,全方位台車で動作を再現させた際のユーザーの位置と角度の予測がグラフィカルに視認できる.
よって適切な位置や方向へ注意を向けることが出来るか,またユーザーの誘導されるであろう位置に障害物はないか,といったことが確認できる.
