\chapter{関連研究}
\section{行動情報から心的状態を推定する研究}
\par
行動情報から心的状態を推定する研究として,人間の心的状態理論Theory of Mind(ToM)をベイズ推論を用いてモデル化するBToM\cite{baker2011bayesian}が存在する.BToMは,環境の状態や人間の心的状態を部分的に観測可能なマルコフ決定過程(POMDP)として表し,環境における人間の行動をベイズ推論に適用することで,人間が観測できていない環境領域についての信念と欲求を共同推定する.人間の信念と欲求を共同推定することで,信念と欲求の相互作用を捉えることができる.また,運転状況から心的状態を推定する研究\cite{darwish2020learning}も存在する.運転者がどのような行動をとることを意図しているかを推定するために,交通状況を部分的に観測可能なマルコフ決定過程として表し,運転者の心的状態をBToMを使ってモデル化している.その結果,自動運転において運転者の意図に沿った動作を実現することができる.また,作業用ロボットに行動情報と心的状態の関係を適用した研究\cite{inbook}も存在する.人間の作業員のアシスタントとして,人間の意図推定できるロボットを導入することで,効率的かつ安全な作業を行うことを目的としている.人間の行動から意図を推定する.人間の心的状態の移り変わりも考慮している.

\section{言語情報から心的状態を推定する研究}
\par
言語情報から心的状態を推定する研究として,発話から信念や欲求,意図を推定する研究\cite{10.1007/978-3-642-02481-8_4}がある.最初に,発話から事象を抽出し,信念として捉える.信念を基に,欲求の候補を複数生成する.生成された欲求の候補に対し,情緒生起手法\cite{2002}を適用し,それぞれの尤度を算出する.願望候補のうち,最も尤度が高い願望を選択し,それを基に意図生成を行う.

\section{単一情報による心的状態推定の問題点}
\par
上記の研究は,いずれも行動情報と発話情報の一方のみを心的状態の推定に用いるため,行動のみを観測できる場合や発話のみを観測できる場合には有効であるが,行動と発話の両方を観測できる場合においては不十分である.
% 例えば,上司と部下がともに行動する人間の心的状態を推定する場面を考える.上司の意思により行動内容が決定された場合,行動情報のみから心的状態を推定する研究では,部下の心的状態を推定することは難しい.
行動によって発話の解釈が変わったり,発話によって行動の解釈が変わることは少なくない.

例えば,

行動情報と発話情報の一方のみの単一情報による心的状態の推定では,行動による発話の解釈の変化や発話による行動の解釈の変化を捉えることができず,行動情報と発話情報の相互作用を考慮して心的状態を推定することができない.
