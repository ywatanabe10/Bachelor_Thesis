\chapter{関連研究}
\section{単一情報から心的状態を推定する研究}
\par
行動情報から心的状態を推定する研究として,人間の心的状態理論Theory of Mind(ToM)をベイズ推論を用いてモデル化するBToM\cite{baker2011bayesian}が存在する.BToMは,環境の状態や人間の心的状態を部分的に観測可能なマルコフ決定過程(POMDP)として表し,環境における人間の行動をベイズ推論に適用することで,人間が観測できていない環境領域についての信念と欲求を共同推定する.人間の信念と欲求を共同推定することで,信念と欲求の相互作用を捉えることができる.

\par
運転状況から心的状態を推定する研究\cite{darwish2020learning}も存在する.運転者がどのような行動をとることを意図しているかを推定するために,交通状況を部分的に観測可能なマルコフ決定過程として表し,運転者の心的状態をBToMを使ってモデル化している.その結果,自動運転において運転者の意図に沿った動作を実現することができる.

\par
また,作業用ロボットに行動情報と心的状態の関係を適用した研究\cite{inbook}も存在する.人間の作業員のアシスタントとして,人間の意図推定できるロボットを導入することで,効率的かつ安全な作業を行うことを目的としている.人間の行動から意図を推定する.人間の心的状態の移り変わりも考慮している.

\section{単一情報から心的状態を推定する研究}
\par
言語情報から心的状態を推定する研究として,発話から信念や欲求,意図を推定する研究\cite{10.1007/978-3-642-02481-8_4}がある.意図を持った対話を行う対話システムの実現のために,信念と欲求を推定し,意図が生成される.BDIモデルは,Beliefモジュール,Desireモジュール,Intentionモジュールから構成される.最初に,Beliefモジュールにおいて発話から事象を抽出し,信念として捉える.Desireモジュールでは,Beliefモジュールにおいて捉えられた信念を基に欲求の候補を複数生成する.また,生成された欲求の候補に対し,情緒生起手法\cite{2002}を適用し,それぞれの尤度を算出し,最も尤度が高いものを欲求とする.Intentionモジュールでは,Desireモジュールにおいて選択された欲求を基に意図生成を行う.

\section{単一情報心的状態推定の問題点}
\par
人間同士の対話では,行動によって発話の解釈が変わったり,発話によって行動の解釈が変わることは少なくない.対話システムと人間との対話においても,行動によって発話の解釈を変えたり,発話によって行動の解釈を変えるために行動情報と発話情報の両方を活用した心的状態の推定が重要である.しかし上記の研究は,いずれも行動情報もしくは発話情報を含む言語情報の一方のみを心的状態の推定に活用したものである.これらの推定は,行動情報のみを観測できる場合や発話情報のみを観測できる場合には有効であるが,行動情報と発話情報の両方を観測できる場合においては不十分である.行動情報と発話情報の一方のみの単一情報による心的状態の推定では,行動による発話の解釈の変化や発話による行動の解釈の変化を捉えることができず,行動情報と発話情報の相互作用を考慮して心的状態を推定することができない.
