\chapter{結論}
本研究ではモデル作成の為の実験を行い,そこで得たデータにより「全方位台車を用いたユーザーの位置・向き 誘導モデル」を作成した.
併せて比較実験を行い,全方位台車による人の誘導の際の回転の有用性を示した.
本研究のモデルによって,誘導の際の全方位台車の位置・向きから,誘導される人の位置・向きを予測することが出来る.
また提案したモデルを元にモデルを図示するツールを作り,それにより誘導経路の修正を行った例を示した.
