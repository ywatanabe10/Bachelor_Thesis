%
%     論文要旨
%
%
\jabst{
\hspace*{1zh}
\par
 本研究の目的は,人間にストレスを与えることの無い,使いやすい対話システムの実現である.単純な質問応答対話を超えた使いやすい対話システムを作るためには,対話相手に合わせて対話戦略を変えることが必要である.対話相手に合わせた対話を行うためには,人間の信念や欲求を推定することが重要である.人間の心的状態を推定する研究では,人間の散策行動から信念や欲求を推定する研究や人間の発話文から信念や欲求,意図を推定する研究がある.しかし既存研究では,
% 人間の散策行動情報と人間の発話情報の両方を用いて心的状態を推定する取り組みは少ないため,
人間の発話文による散策行動の解釈や散策行動による人間の発話の解釈を捉える方法は未だ確立されていない.本論文では,散策行動情報と人間の発話情報の両方を活用し,信念と欲求を逐次的に推定するシステムMultimodal Inference of Mind SCAIN (MIoM SCAIN)を提案する.MIoM SCAINは,行動情報と発話情報の両方を活用して信念と欲求を推定することで,人間の発話内容に依存した散策行動の解釈の決定や散策行動による人間の発話文の解釈を決定する.MIoM SCAINでは,独自に作成したデータセットを利用し推定システムを構築することで,人間の信念と欲求の推定を行うことができる.また,本論文では,散策行動情報と人間の発話情報の両方を活用した人間の信念と欲求の推定が有効であるかを評価する実験を行った.評価実験の結果,MIoM SCAINの推定性能が散策行動情報と人間の発話情報の一方のみから信念と欲求を推定するシステムの推定性能を上回り,散策行動情報と人間の発話情報の両方を活用した信念と欲求の推定が有効であることが示された.
}
\makejabstract
