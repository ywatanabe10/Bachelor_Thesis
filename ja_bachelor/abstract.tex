%
%     論文要旨
%
%
\jabst{
\hspace*{1zh}
\par
 本研究の目的は,人間にストレスを与えることの無い,人間と長期的に共存可能な対話システムの実現である.対話システムが人間の心的状態を考慮せず対話を行うと,相手に不自然さやストレスを与えることがあるため,対話相手に合わせて臨機応変に対話戦略を変えることが必要である.対話相手に合わせた対話を行うためには,人間の心的状態を推定することが重要である.人間の心的状態を推定する研究では,人間の行動から心的状態を推定する研究や人間の発話から心的状態を推定する研究がある.しかし既存研究では,行動情報と発話情報の両方を用いて心的状態を推定することができていないため,発話による行動の解釈の変化や行動による発話の解釈の変化を捉えることができていない.

\par
 本論文では,行動情報と発話情報の両方を活用し,心的状態の一部である信念と欲求を逐次的に推定するシステムMultimodal Inference of Mind SCAIN (MIoM SCAIN)を提案する.MIoM SCAINは,行動情報と発話情報の両方を活用して信念と欲求を推定することで,発話による行動の解釈の変化や行動による発話の解釈の変化を捉える.本研究では,独自に作成したデータセットを利用し,MIoM SCAINによる信念と欲求の推定を行い,行動情報と発話情報の両方を活用した信念と欲求の推定が有効であるかを評価する実験を行った.本実験の結果,MIoM SCAINの推定性能が行動情報と発話情報の一方のみから信念と欲求を推定するシステムの推定性能を上回り,行動情報と発話情報の両方を活用した信念と欲求の推定が有効であることが示された.
}
\makejabstract
