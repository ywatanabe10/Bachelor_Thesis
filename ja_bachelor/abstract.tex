%
%     論文要旨
%
%
\jabst{
\hspace*{1zh}
\par
 本研究の目的は,人間にストレスを与えることの無い,人間と長期的に共存可能な対話システムの実現である.対話システムは,人間の心的状態を考慮せず発話解釈や発話生成を行い,相手に不自然さやストレスを与えることがあるため,対話相手に合わせて臨機応変に発話解釈や発話生成を行うことが必要である.対話相手に合わせた対話を行うためには,人間の心的状態を推定することが重要である.人間の心的状態を推定する研究では,人間の行動から心的状態を推定する研究や人間の発話から心的状態を推定する研究がある.しかし既存研究では,行動情報と発話情報の両方を用いて心的状態を推定することができていないため,発話による行動の解釈の変化や行動による発話の解釈の変化を捉えることができていない.

\par
 本論文では,行動情報と発話情報の両方を活用した心的状態推定システムMultimodal Inference of Mind (MIoM)を提案する.MIoMは,行動情報と発話情報の両方を活用して心的状態を推定することで,発話による行動の解釈の変化や行動による発話の解釈の変化を捉える.本実験では,本研究で作成したデータセットを利用し,MIoMによる心的状態の推定を行い,行動情報と発話情報の両方を活用した心的状態の推定が有効であるかを評価した.本実験の結果,MIoMの推定性能が単一情報のみから心的状態を推定するシステムの推定性能を上回り,行動情報と発話情報の両方を活用した心的状態の推定が有効であることが示された.
}
\makejabstract
