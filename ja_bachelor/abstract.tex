%
%     論文要旨
%
%
\jabst{
\hspace*{1zh}
今後一般家庭や公共施設では荷物を運ぶ搬送用ロボットが普及すると思われる.
人間のいる空間で動作するロボットの移動機構として,並進移動や回転が出来て小さなスペースでも素早く移動できることが求められる.
そのような要求を最適に満たすのは,全方位台車と呼ばれるホイールを等角度間隔で3個以上配置した移動機構である.
\hspace*{1zh}
\par
 ロボットによる人の誘導の研究は行われてきたが,位置と向きを同時に誘導している例は少ない.
またそのような研究でも手によるジェスチャーなどの誘導以外の用途に役立たない汎用性の低いデバイスが必要である.
本稿では「全方位台車を用いたユーザーの位置・向き 誘導モデル」を提案する.
本モデルを用いると,全方位台車の自由な移動及び回転のみによって人の位置と向きを同時に誘導することが出来る.
\hspace*{1zh}
\par
 今回のモデルは,全方位台車で人間が他の人間を誘導する動きを再現させ,その際の全方位台車及び人間の位置関係や向きを分析することにより作成した.
具体的には,まず人間に位置および向きの誘導を行わせその動きのデータをモーションキャプチャで取得した.
次にそのデータから位置と向きの情報を抽出しそれらの動きを全方位台車に再現させ,全方位台車による案内を被験者に受けさせる実験を行った.
実験における全方位台車と被験者の位置および向きの関係を調べ,モデル化を行った.
\hspace*{1zh}
\par
 モデルの有効性を示すため,作成した誘導動作において予測される人間の位置と角度を
図示するツールを作成した.



}
\makejabstract
