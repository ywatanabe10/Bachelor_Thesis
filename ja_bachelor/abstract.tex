%
%     論文要旨
%
%
\jabst{
\hspace*{1zh}
% 今後一般家庭や公共施設では荷物を運ぶ搬送用ロボットが普及すると思われる.
% 人間のいる空間で動作するロボットの移動機構として,並進移動や回転が出来て小さなスペースでも素早く移動できることが求められる.
% そのような要求を最適に満たすのは,全方位台車と呼ばれるホイールを等角度間隔で3個以上配置した移動機構である.

 本研究の目的は、人間にストレスを与えること無い、人間と長期的に共存可能な対話システムの実現である。しかし、対話システムは相手の心的状態を考慮することなく対話をすることがあり、対話相手に不自然さやストレスを与えることがあるため、対話システムは相手の心的状態を推定することが重要である。そこで、本論文では心的状態推定アルゴリズムMIoMを提案する。MIoMは、行動情報と言語情報の両方を用いて心的状態を推定することにより、対話システムが人間に与える不自然さやストレスを軽減することの手助けとなる。本実験では、本研究で作成した、行動情報や言語情報を含むデータセットを用い、MIoMによる心的状態の推定を行った。人間評価とアブレーションスタディにより、行動情報と言語情報の両方を心的状態の推定に活用することが有効であることが示された。

}
\makejabstract
