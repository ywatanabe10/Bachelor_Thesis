\chapter{背景}
本章では問題設定および作成すべきモデルについて記述する.

\section{問題設定}

\subsection{全方位台車について}
全方位台車とは,ホイールを等角度間隔で3個以上配置した移動機構を持つ車両のことである.
全方位台車は並進移動や回転により小さなスペースても素早く移動できるので,
今後一般家庭や公共施設にも搬送なとの用途で普及すると思われる.
例えば今井らは病院内ロボット搬送システムの開発を研究しているが,
ここで用いられているロボットには全方位移動機構が採用されている\cite{Imai}.
\par
全方位台車の普及を受けて,動作に関する研究が多く行われている.
例えば,L.Huangらは90度間隔でホイールを4輪持つ全方位台車を設計した.
また動作アルゴリズムを作成し,直線や円の軌跡をトレースして動けるか動作検証が行われた\cite{L}.
多田隈らは全方位台車を任意の方向へ任意の回転速度で移動させるための,ホイールの回転速度
決定に関する研究が行われた\cite{Tada}.
本稿の全方位台車の動作はその回転速度決定の行列式を元に設計されている.
\par
全方位台車の位置座標取得は大きく分けて,全方位台車に搭載されたデバイスにより行う物と外部のシステムから取得するものがある.
全方位台車に搭載するものとしては,レーザーレンジファインダによるSLAMといったものがある.
外部のシステムでは,モーションキャプチャが一般的であると思われる.

\subsection{全方位台車による人の位置向き誘導の必要性}
\label{yuudou}
人間のタスクを全方位台車がサポートするに当って,全方位台車により人間に場所や方向を指示する必要がある場面が考えられる.
例えば荷物を置き場まで運ぶタスクにおいては,人間に運ぶべき荷物やそれを置く場所・方向を指示することが求められる.
このようなタスクにおいて人間をサポートする場合,全方位台車により人の位置・向きの誘導が行える必要がある.

\section{作成すべきモデル}
\ref{yuudou}において説明した全方位台車による人の位置・向き誘導を行うためには,
全方位台車の位置・向きに対して人がどのような位置・向きに誘導されるかを示す
モデルが必要となる.
本稿ではそのモデルを全方位台車による人の位置・向き誘導の実験を通して作成する.
また,作成したモデルに従い全方位台車の動作に対して予測される人の動作を
図示するツールを作る.
これにより全方位台車の誘導の動作を改良することが出来る.